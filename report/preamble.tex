\documentclass[12pt,a4paper]{article}
\usepackage[margin=1in]{geometry}
%\usepackage{ghw}
\usepackage[utf8]{inputenc}
\usepackage{amsmath}
\usepackage{amsfonts}
\usepackage{amssymb}
\usepackage{array} % beatiful latex table
%\usepackage[url=false]{biblatex}
\usepackage{caption}
\usepackage{color}
\usepackage{graphicx}
\usepackage{epstopdf} %This is not necessary if I use pdf instead of eps
\usepackage{multirow}
\usepackage[round]{natbib}
\usepackage{subcaption} %for side by side figures
\usepackage{rotating}
\usepackage{dcolumn} % for align parameter in stargazer package
\usepackage{longtable}
\usepackage{tabu}
\usepackage{threeparttable}
\usepackage[table]{xcolor} %for table colors
%\usepackage{floatrow} % file specific packages
%\usepackage[capposition=top]{floatrow} % add notes under caption for figures
\usepackage[doublespacing]{setspace} % set double space withtout doubing the space of tables

%%% Math Symbols %%%%%%%%%%%%%%%%%%%%%%%%%%%%%%%%%%%%%%%%%%%%%%%%%%%%%%%%%%%%%%%
% Dependent and Independent Symbols
% ref: http://tex.stackexchange.com/questions/174118/not-independent-sign-in-latex 
\makeatletter
\newcommand*{\indep}{%
	\mathbin{%
		\mathpalette{\@indep}{}%
	}%
}
\newcommand*{\nindep}{%
	\mathbin{%                   % The final symbol is a binary math operator
		\mathpalette{\@indep}{\not}% \mathpalette helps for the adaptation
		% of the symbol to the different math styles.
	}%
}
\newcommand*{\@indep}[2]{%
	% #1: math style
	% #2: empty or \not
	\sbox0{$#1\perp\m@th$}%        box 0 contains \perp symbol
	\sbox2{$#1=$}%                 box 2 for the height of =
	\sbox4{$#1\vcenter{}$}%        box 4 for the height of the math axis
	\rlap{\copy0}%                 first \perp
	\dimen@=\dimexpr\ht2-\ht4-.2pt\relax
	% The equals symbol is centered around the math axis.
	% The following equations are used to calculate the
	% right shift of the second \perp:
	% [1] ht(equals) - ht(math_axis) = line_width + 0.5 gap
	% [2] right_shift(second_perp) = line_width + gap
	% The line width is approximated by the default line width of 0.4pt
	\kern\dimen@
	{#2}%
	% {\not} in case of \nindep;
	% the braces convert the relational symbol \not to an ordinary
	% math object without additional horizontal spacing.
	\kern\dimen@
	\copy0 %                       second \perp
} 
\makeatother
%
% plim:
\newcommand{{\plim}}{{\ \text{p lim} \ }}
% convergence:
\newcommand{{\convc}}{{\ \buildrel c \over \longrightarrow \ }}
\newcommand{{\convd}}{{\ \buildrel d \over \longrightarrow \ }}
\newcommand{{\convp}}{{\ \buildrel p \over \longrightarrow \ }}
\newcommand{{\convm}}{{\ \buildrel M \over \longrightarrow \ }}
\newcommand{{\convas}}{{\ \buildrel a.s. \over \longrightarrow \ }}
\newcommand{{\eqld}}{{\ \buildrel LD \over = \ }}
%